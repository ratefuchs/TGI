\documentclass{beamer}
\usepackage[utf8]{inputenc}
\usepackage[ngerman]{babel}

\usetheme[deutsch]{KIT}
\author{Simon Bischof (simon.bischof2@student.kit.edu)}
\title{Tutorium Theoretische Grundlagen der Informatik}
\subtitle{Simon Bischof}
\institute{Institut f\"{u}r Kryptographie und Sicherheit}
\TitleImage[scale=0.7]{tmaschine.png}

\newcommand{\F}{\Sigma^*}
\newcommand{\N}{\ensuremath \mathbb{N}}
\newcommand{\R}{\ensuremath \mathbb{R}}
\renewcommand{\P}{\ensuremath \mathcal{P}}
\newcommand{\NP}{\ensuremath \mathcal{NP}}
\newcommand{\NPC}{\ensuremath \mathcal{NP-C}}

\begin{document}
\shorthandoff{"}
\begin{frame}
\maketitle
\end{frame}

\begin{frame}
\frametitle{\LARGE Geh wählen! -- Nur noch heute!}
Heute ist letzter Tag der U-Wahlen und der VS-Urabstimmung.\\
Wählen könnt ihr noch bis
\begin{itemize}
\item 15:00 Uhr an den meisten Urnen
\item 16:00 Uhr an der Mensa-Urne
\end{itemize}
\end{frame}

\begin{frame}
\frametitle{Zum Übungsblatt}
\begin{itemize}
\item für $L\in\NP$ und polynomielle Reduktion: immer hinschreiben, dass in Polyzeit möglich
\item bei Reduktion $w\in A\Leftrightarrow f(w)\in B$ begründen
\item für $L\in\NPC$ immer auch $L\in\NP$ nötig
\end{itemize}
\end{frame}

\begin{frame}
\frametitle{Folien jetzt auf Github}
Die Folien für mein Tut und die Tutblätter sind jetzt auf Github unter\\
\textcolor{blue}{\underline{\url{https://github.com/ratefuchs/TGI}}} zu finden.
\end{frame}
\end{document}