\documentclass{beamer}
\usepackage[utf8]{inputenc}
\usepackage[ngerman]{babel}

\usetheme[deutsch]{KIT}
\author{Simon Bischof (simon.bischof2@student.kit.edu)}
\title{Tutorium Theoretische Grundlagen der Informatik}
\subtitle{Simon Bischof}
\institute{Institut f\"{u}r Kryptographie und Sicherheit}
\TitleImage[scale=0.7]{tmaschine.png}

\newcommand{\F}{\ensuremath \mathbb{F}}
\newcommand{\N}{\ensuremath \mathbb{N}}
\newcommand{\R}{\ensuremath \mathbb{R}}
\newcommand{\E}{\ensuremath \mathbb{E}}
\renewcommand{\P}{\ensuremath \mathcal{P}}
\newcommand{\NP}{\ensuremath \mathcal{NP}}
\newcommand{\NPC}{\ensuremath \mathcal{NP-C}}

\begin{document}
\shorthandoff{"}
\begin{frame}
\maketitle
\end{frame}

\begin{frame}
\frametitle{Erinnerung}
\begin{itemize}
\item Abholung Übungsblatt 7 bei den Übungsleitern
\item Hauptklausur: 22.02., 8:00 Uhr
\item Anmeldung nicht vergessen (bis 15.02.!)
\item Nachklausur: 10.04., 11:30 Uhr
\item Klausur geht 60 min
\item Es gibt 60 Punkte, 20 sind zum Bestehen hinreichend
\item Keine Hilfsmittel erlaubt
\end{itemize}
\end{frame}

\begin{frame}
\frametitle{Bei Fragen}
\begin{itemize}
\item Kommilitonen fragen
\item mir eine E-Mail schreiben (simon.bischof2@student.kit.edu)
\item Mail an die Übungsleiter
\end{itemize}
(möglichst in dieser Reihenfolge)
\end{frame}

\begin{frame}
\frametitle{Zum 7. Übungsblatt}
\begin{itemize}
\item Achtung beim Hammingcode, falls Prüfmatrix nicht in systematischer Form vorliegt (nur dort ist das Syndrom die Binärcodierung der Fehlerstelle)
\item "2-Bit-Fehler nicht falsch korrigiert" bedeutet nicht unbedingt, dass dann korrekt dekodiert wird; es könnte auch sein, dass keine Korrektur unternommen wird (da nicht eindeutig).
\end{itemize}
\end{frame}
\end{document}