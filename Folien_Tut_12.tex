\documentclass{beamer}
\usepackage[utf8]{inputenc}
\usepackage[ngerman]{babel}

\usetheme[deutsch]{KIT}
\author{Simon Bischof (simon.bischof2@student.kit.edu)}
\title{Tutorium Theoretische Grundlagen der Informatik}
\subtitle{Simon Bischof}
\institute{Institut f\"{u}r Kryptographie und Sicherheit}
\TitleImage[scale=0.7]{tmaschine.png}

\newcommand{\F}{\Sigma^*}
\newcommand{\N}{\ensuremath \mathbb{N}}
\newcommand{\R}{\ensuremath \mathbb{R}}
\newcommand{\E}{\ensuremath \mathbb{E}}
\renewcommand{\P}{\ensuremath \mathcal{P}}
\newcommand{\NP}{\ensuremath \mathcal{NP}}
\newcommand{\NPC}{\ensuremath \mathcal{NP-C}}

\begin{document}
\shorthandoff{"}
\begin{frame}
\maketitle
\end{frame}

\begin{frame}
\frametitle{Organisatorisches}
\begin{itemize}
\item Abgabe ÜB 6 am Mo., 28.01
\item Veröffentlichung ÜB 7 schon am Mo., 28.01
\item Abgabe von ÜB 7 schon Fr., 01.02.
\item Blatt 7: Nur 2 Aufgaben werden gewertet
\item Zusätzlich Wiederholungsaufgaben: werden nicht gewertet. Ich werde diese Aufgaben dennoch korrigieren.
\end{itemize}
\end{frame}

\begin{frame}
\frametitle{VL letzte Woche (VL-Folien/Skript)}
\begin{itemize}
\item 3SAT $\leq_p$ HAMILTON-PATH
\item Fixed parameter tractability
\item Average case complexity
\item Impaliazzos Welten
\item Counting Classes und Probabilistische Klassen:\\
$\#\P,\mathcal{R},co\mathcal{R},\mathcal{ZPP},\mathcal{PP},\mathcal{BPP}$ 
\end{itemize}
\end{frame}

\begin{frame}
\frametitle{Informationstheorie}
\end{frame}

\begin{frame}
\frametitle{Rechenregeln für $\log$ und Wahrscheinlichkeiten}
\begin{itemize}
\item $\log(a\cdot b)=\log a+\log b$, $\log(a^b)=b\log(a)$, insbesondere $\log\frac{1}{a}=-log(a)$
\item Hier: $0\cdot\log 0:=0,0\cdot\log \frac{0}{0}=0,a\cdot\log \frac{a}{0}=\infty$ $(a>0)$\pause
\item Sind $X_1,X_2,\ldots,X_n$ stochastisch unabhängige ZV, so ist\\
$p(X_1=x_1,X_2=x_2,\ldots,X_n=x_n)=p(X_1=x_1)\cdot p(X_2=x_2)\cdot \ldots\cdot p(X_n=x_n)$
\item Erwartungswert: $\E(X)=\sum\limits_{x\in X}x\cdot p(x)$\pause
\item Bedingte Wahrscheinlichkeit: $p(y|x)=\frac{p(x,y)}{p(x)}$
\end{itemize}
\end{frame}

\begin{frame}
\frametitle{Eigenschaften von Information}
Information $I_p$ für Zeichen der Wahrscheinlichkeit (WK) $p$
\begin{itemize}
\item $I_p\geq 0$
\item $I_1=0$
\item $I$ ist stetig
\item Sind zwei Ereignisse mit WK $p_1,p_2$ stochastisch unabhängig, dann $I_{(p_1\cdot p_2)}=I_{p_1}+I_{p_2}$
\end{itemize}
\end{frame}

\begin{frame}
\frametitle{Quellen, Information, Entropie}
\begin{itemize}
\item Eine Quelle sendet Zeichen mit einer gewissen WK aus.\pause
\item Sie heißt gedächtnislos, falls die WK für ein Zeichen nicht von den vorigen Zeichen abhängt.\pause
\item $I_p:=\log_b\frac{1}{p} [=-\log_b p]$
\item Hier immer $b=2$, dadurch Einheit "bit"\pause
\item Entropie: $H(X)=\sum\limits_{x\in X}p(x)\log\frac{1}{p(x)}=\E I(x)$\pause
\item Für Basis $b$ gilt: $H_b(X) = \log_b 2\cdot H(X)$\pause
\item maximale Entropie bei $n$ Zeichen: $\log n$; genau dann wenn alle Zeichen gleich wahrscheinlich sind (WK $\frac{1}{n}$)
\end{itemize}
\end{frame}

\begin{frame}
\frametitle{Gemeinsame und bedingte Entropie}
\begin{itemize}
\item Gemeinsame Entropie: $H(X,Y)=\sum\limits_{x\in X}\sum\limits_{y\in Y}p(x,y)\log\frac{1}{p(x,y)}$\pause
\item Bedingte Entropie: $H(Y|X)=\sum\limits_{x\in X}H(Y|X=x)=-\sum\limits_{x\in X}p(x)\sum\limits_{y\in Y}p(y|x)\log(p(y|x))=-\sum\limits_{x\in X,y\in Y}p(x,y)\log(p(y|x))$
\pause
\item Kettenregel: $H(X,Y)=H(X)+H(Y|X)=H(Y)+H(X|Y)$
\item[$\Rightarrow$] $H((X,Y)|Z)=H(X|Z)+H(Y|X,Z)$
\item  Achtung: Es kann sein dass $H(X|Y)\neq H(Y|X)$!
\end{itemize}
\end{frame}

\begin{frame}
\frametitle{Datenübertragung}
Sei ein Kanal gegeben, der ein Symbol von $X$ nach $Y$ überträgt, und evtl. nicht fehlerfrei ist.
\begin{itemize}
\item $H(X,Y)$ heißt Totalinformation
\item $H(X|Y)$ heißt Äquivokation
\item $H(Y|X)$ heißt Fehlinformation/Irrelevanz\pause
\item Transinformation: $I(X;Y):=H(X)-H(X|Y)=H(Y)-H(Y|X)$
\end{itemize}
\end{frame}

\begin{frame}
\frametitle{Präfix Codes}
\begin{itemize}
\item Sei $Q$ ein Alphabet und $f:Q\to \F$ eine Kodierung.\pause
\item Achtung: Evtl. ist die Dekodierung nicht eindeutig!\pause
\item Falls für alle Codewörter $c_1c_2\ldots c_n\in f(Q)$ und für $k<n$ das Wort $c_1c_2\ldots c_k$ kein Codewort ist, so heißt die Kodierung Präfix-Code.
\item Präfix-Codes sind immer eindeutig.
\end{itemize}
\end{frame}

\begin{frame}
\frametitle{Die Huffman-Kodierung}
\begin{itemize}
\item Die Huffman-Kodierung ist ein für gedächtnislose Quellen optimaler Präfix-Code.
\vspace{1cm}\pause
\item Konstruktion: Für jedes Zeichen ein eigener Knoten. Häufigkeit dazuschreiben.\pause
\item Dann schrittweise einen Baum erstellen:\\
Zwei (Wurzel-)Knoten aussuchen, dass Summe der Häufigkeiten minimal wird.
\item Füge dann einen neuen Knoten als Vater dieser beiden Knoten hinzu und schreibe diese Summe dazu.\pause
\item Am Ende: Wege nach links: 0, nach rechts: 1
\end{itemize}
\end{frame}
\end{document}