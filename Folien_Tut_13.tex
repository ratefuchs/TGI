\documentclass{beamer}
\usepackage[utf8]{inputenc}
\usepackage[ngerman]{babel}

\usetheme[deutsch]{KIT}
\author{Simon Bischof (simon.bischof2@student.kit.edu)}
\title{Tutorium Theoretische Grundlagen der Informatik}
\subtitle{Simon Bischof}
\institute{Institut f\"{u}r Kryptographie und Sicherheit}
\TitleImage[scale=0.7]{tmaschine.png}

\newcommand{\F}{\ensuremath \mathbb{F}}
\newcommand{\N}{\ensuremath \mathbb{N}}
\newcommand{\R}{\ensuremath \mathbb{R}}
\newcommand{\E}{\ensuremath \mathbb{E}}
\renewcommand{\P}{\ensuremath \mathcal{P}}
\newcommand{\NP}{\ensuremath \mathcal{NP}}
\newcommand{\NPC}{\ensuremath \mathcal{NP-C}}

\begin{document}
\shorthandoff{"}
\begin{frame}
\maketitle
\end{frame}

\begin{frame}
\frametitle{Organisatorisches}
\begin{itemize}
\item ÜB7: Abgabe am 1.2., 12:00 Uhr; Abholung ab 8.2. (im Tut oder danach bei den Übungsleitern) 
\item Hauptklausur: 22.02., 8:00 Uhr
\item Anmeldung ab sofort bis 15.2.
\item Nachklausur: 10.04., 11:30 Uhr
\item Klausur geht 60 min
\item Es gibt 60 Punkte, 20 sind zum Bestehen hinreichend
\item Keine Hilfsmittel erlaubt
\end{itemize}
\end{frame}

\begin{frame}
\frametitle{Präfix Codes}
\begin{itemize}
\item Sei $Q$ ein Alphabet und $f:Q\to \{0,1\}^*$ eine Kodierung.\pause
\item Achtung: Evtl. ist die Dekodierung nicht eindeutig!\pause
\item Falls für alle Codewörter $c_1c_2\ldots c_n\in f(Q)$ und für $k<n$ das Wort $c_1c_2\ldots c_k$ kein Codewort ist, so heißt die Kodierung Präfix-Code.
\item Präfix-Codes sind immer eindeutig.\pause
\item Für einen Präfixcode gilt: Die mittlere Codewortlänge ist größer oder gleich der Entropie der Quelle
\end{itemize}
\end{frame}

\begin{frame}
\frametitle{Shannon-Fano}
\begin{itemize}
\item Shannon-Fano konstruiert einen (nicht immer optimalen) Präfix-Code.
\vspace{1cm}\pause
\item Sortiere die vorkommenden Symbole nach ihrer Häufigkeit.\pause
\item Bestimme nun den Punkt, an dem die Reihe aus Symbolen aufgeteilt
werden muss, so dass die aufsummierten Wahrscheinlichkeiten der beiden
entstehenden Gruppen möglichst gleich sind.\pause
\item Hänge nun die beiden entstehenden Gruppen als Blätter an eine neue
Wurzel.\pause
\item Verfahre nun rekursiv: Ersetze die Gruppen jeweils durch den Baum der
beim Anwenden des Verfahrens auf sie jeweils entsteht, solange, bis alle
Blätter einzelne Symbole sind.\pause
\item Am Ende: Wege nach links: 0, nach rechts: 1
\end{itemize}
\end{frame}

\begin{frame}
\frametitle{Kanalkodierungen}
\begin{itemize}
\item Daten werden über einen Kanal geschickt\pause
\item Kanal verändert (zufällig) einzelne Bits\pause
\item Wie sichere ich die Daten?
\end{itemize}
\end{frame}

\begin{frame}
\frametitle{Grundlagen}
\begin{itemize}
\item Hammingdistanz für $x,y \in \{0,1\}^n$: $d(x,y):=\sum\limits_{i=1}^n  (1-\delta_{x_i,y_i})=\#\{i=1,\ldots,n|x_i\neq y_i\}$\pause
\item Hammingkugel um $x$ mit Radius $\varrho$: $B_\varrho(x):=\{y\in\{0,1\}^n|d(x,y)\leq \varrho\}$\pause
\item Maximum-Likelyhood-Decoding: Dekodiere empfangenes Wort $y$ als Codewort $x$ mit $d(x,y)$ minimal.\pause
\item Rate für Code mit $M$ Wörtern der Länge $n$: $R=\frac{\log_2 M}{n}$\pause
\item Mit Maximum-Likelyhood-Decoding werde gesendetes $x_i$ mit WK $P_i$ falsch dekodiert. Wahrscheinlichkeit für falsche Dekodierung: $P_C=\frac{1}{M}\sum\limits_{i=1}^M P_i$
\end{itemize}
\end{frame}

\begin{frame}
\frametitle{Kanalkodierungen}
Hier gibt es vor allem zwei Möglichkeiten:\pause
\begin{itemize}
\item Block-Codes: Codeworte gleicher Länge, unabhängige Kodierung von aufeinanderfolgenden Blöcken
\item Faltungs-Codes: Codeworte evtl. beliebig lang, Kodierung abhängig vom Vorgeschehen\pause
\end{itemize}
In der VL: Nur Block-Codes
\end{frame}

\begin{frame}
\frametitle{Eigenschaften von Block-Codes}
\begin{itemize}
\item Code immer über endl. Alphabet $Q$. Oft: $Q=\{0,1\}$.
\item Der Block-Code ist eine Teilmenge $C\subseteq Q^n$ für festes $n\in\N$.
\item Code heißt für $\#Q=1$ trivial (nur ein Codewort!), $\#Q=2$ binär, $\#Q=3$ terniär, \ldots\pause
\item Minimaldistanz $m(C):=\min\limits_{c_1,c_2\in C,c_1\neq c_2} d(c_1,c_2)$
\item Rate $R(C):=\frac{\log(\#C)}{\log(\#Q^n)}=\frac{\log(\#C)}{n\cdot\log(\#Q)}$
\item Ein Code $C$ mit $m(C)$ ungerade heißt perfekt, falls für alle $y\in Q^n$ genau ein $x\in C$ existiert mit $d(x,y)\leq \frac{m(C)-1}{2}$\pause
\item Ein Block-Code $C$ kann entweder bis zu $m(C)-1$ Fehler erkennen oder bis zu $\lfloor\frac{m(C)-1}{2}\rfloor$ Fehler korrigieren.
\end{itemize}
\end{frame}

\begin{frame}
\frametitle{Lineare Codes}
Normalerweise: Beliebiger endlicher Körper $\F_q$. Hier immer: $\F=\F_2=\{0,1\}$.
\begin{itemize}
\item Ein linearer $[n,k]$-Block-Code $C$ ist ein Untervektorraum von $\F$ der Dimension $k$.\pause
\item Hamming-Gewicht: $wgt(x):=d(x,0)$. Es ist außerdem $d(x,y)=wgt(x-y)$.\pause
\item Beschreibe $C$ als Kern einer $\F^{(n-k)\times n}$-Matrix (Parity-Check- oder Prüf-Matrix): $C=Kern(H)=\{x\in\F^n|Hx=0\}$
\item \ldots oder als Bild einer $\F^{n\times k}$-Matrix (Generatormatrix): $C=Bild(G)=\{Gx|x\in\F^k\}$.\pause
\item $s=Hx$ heißt das Fehlersyndrom von $x\in\F^n$.\pause
\item Für gegebenes $s$ heißt (falls eindeutig) das $e\in\F^n$ mit
$wgt(e) = min\{wgt(x)|x \in \F^n\setminus\{0\}, Hx=s\}$ der Coset-Leader von $s$.
\end{itemize}
\end{frame}

\begin{frame}
\frametitle{Hamming-Codes}
\begin{itemize}
\item Hamming-Codes sind $[2^k - 1, 2^k - k - 1]$-Codes, für die je
zwei Spalten der Prüfmatrix linear unabhängig sind.\pause
\item Es gilt $R(C)=\frac{2^k-k-1}{2^k-1}=1-\frac{k}{2^k-1}$ und $m(C)=3$.
\item Hamming-Codes sind perfekt.
\end{itemize}
\end{frame}
\end{document}