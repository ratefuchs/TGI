\documentclass{beamer}
\usepackage[utf8]{inputenc}
\usepackage[ngerman]{babel}

\usetheme[deutsch]{KIT}
\author{Simon Bischof (simon.bischof2@student.kit.edu)}
\title{Tutorium Theoretische Grundlagen der Informatik}
\subtitle{Simon Bischof}
\institute{Institut f\"{u}r Kryptographie und Sicherheit}
\TitleImage[scale=0.7]{tmaschine.png}

\newcommand{\F}{\Sigma^*}
\newcommand{\N}{\ensuremath \mathbb{N}}
\newcommand{\R}{\ensuremath \mathbb{R}}
\renewcommand{\P}{\ensuremath \mathcal{P}}
\newcommand{\NP}{\ensuremath \mathcal{NP}}

\begin{document}
\shorthandoff{"}
\begin{frame}
\maketitle
\end{frame}

\begin{frame}
\frametitle{Organisatorisches}
\begin{itemize}
\item Literaturempfehlung: Michael Sipser, Introduction to the Theory of Computation; in der KIT-Bibliothek Süd im Bereich 1.0, in der Informatik-Bibliothek im Bereich D.Sip.
\item Tutorium am 21.12.2012 findet normal statt
\item Eulenfest: Mittwoch, 19.12.2012 ab 20:30 beim Infobau
\end{itemize}
\end{frame}

\begin{frame}
\frametitle{Optimalität der Kolmogorov-Komplexität}
\begin{itemize}
\item Sei $p$ eine Beschreibungssprache und $K_p$ die zugehörige Beschreibungskomplexität.
\item Dann existiert ein $c$ mit $K(w) \leq K_p(w) + c \quad (w\in\{0,1\}^*)$.
\end{itemize}
\end{frame}

\begin{frame}
\frametitle{nichtkomprimierbare Strings}
\begin{itemize}
\item Für alle $n\in\N_0$ gibt es nichtkomprimierbare Strings der Länge $n$.\pause
\item Fast alle Strings sind nichtkomprimierbar
\item Zufällige Strings sind mit hoher Wahrscheinlichkeit nichtkomprimierbar\pause
\item Erinnerung: Die Menge der nichtkomprimierbaren Strings ist nicht rekursiv aufzählbar
\end{itemize}
\end{frame}

\begin{frame}
\frametitle{Komplexitätstheorie}
\end{frame}

\begin{frame}
\frametitle{Laufzeit von TM}
\begin{itemize}
\item Die TM $M$ halte für alle Eingaben.
\item $f(n):=\max\limits_{|w|=n} (\text{Anzahl der Berechnungsschritte von $M$ bei Eingabe $w$})$ nennt man Laufzeit der TM $M$.
\end{itemize}
\end{frame}

\begin{frame}
\frametitle{Die O-Notation}
Seien $f,g:\N\to\R^+$. Wir schreiben
\begin{itemize}
\item $f\in O(g(n))$ wenn $\exists c,n_0\in\N \forall n\geq n_0: f(n)\leq c\cdot g(n)$\pause
\item $f\in o(g(n))$ wenn $\forall c\in\R^+ \exists n_0\in\N_0 \forall n\geq n_0: f(n)< c\cdot g(n)$
\item Andere Formulierung: $f\in o(g(n))$ wenn $\lim\limits_{n\to\infty}\frac{f(n)}{g(n)}=0$.
\end{itemize}
\end{frame}

\begin{frame}
\frametitle{Das Speed-Up-Theorem}
\begin{itemize}
\item Zu jeder TM gibt es eine sprachäquivalente TM, die um einen konstanten Faktor schneller ist.
\end{itemize}
\end{frame}

\begin{frame}
\frametitle{Komplexitätsklassen}
\begin{itemize}
\item Für $t:\N\to\N$ ist $TIME(t(n)):=\{L|L\text{ ist entscheidbar durch eine TM, die bei Eingabelänge $n$} \newline \text{ $O(t(n))$ Schritte benötigt.}\}$
\item Für eine Mehrband-TM $M$ mit Laufzeit $O(t(n))$ ist $L(M)\in TIME(O(t^2(n)))$.
\end{itemize}
\end{frame}

\begin{frame}
\frametitle{Laufzeit einer NTM}
\begin{itemize}
\item Sei $M$ nichtdeterministische TM, die immer hält und $P(w)$ die Menge der Berechnungspfade bei Eingabe $w$.
\item $f(n):=\max\limits_{|w|=n}\text{ } \min\limits_{p \in P(w)}(\text{Länge von $p$})$
\end{itemize}
\end{frame}

\begin{frame}
\frametitle{Verifizierer}
\begin{itemize}
\item Ein Verifizierer für eine Sprache $A$ ist ein Algorithmus V mit $A=\{w|\exists c\in\F : V\text{ akzeptiert } (w,c)\}$
\item Wenn die Laufzeit von $V$ polynomial in $|w|$: $A$ ist polynomial verifizierbar
\item $c$ nennt man Zeuge.
\end{itemize}
\end{frame}

\begin{frame}
\frametitle{Die Klasse $\P$}
\begin{itemize}
\item $\P:=\bigcup\limits_{k\in\N} TIME(n^k)$
\item effizient lösbare Sprachen
\end{itemize}
\end{frame}

\begin{frame}
\frametitle{Die Klasse $\NP$}
\begin{itemize}
\item $\NP$: polynomiell verifizierbare Sprachen\pause
\item $NTIME(t(n)):=\{L|L\text{ wird von einer NTM in Zeit $O(t(n))$ akzeptiert}\}$
\item $\NP=\bigcup\limits_{k\in\N} NTIME(n^k)$\pause
\item $\P\subseteq\NP$
\item Großes Problem der theoretischen Informatik: Ist $\P=\NP$? (Vermutung: nein!)
\end{itemize}
\end{frame}

\begin{frame}
\frametitle{Probleme in $\P$ bzw. $\NP$}
In $\P$:
\begin{itemize}
\item PATH
\item RELPRIME
\item EULER-KREIS
\item COMPOSITE
\end{itemize}
In $\NP$ (aber vermutlich nicht in $\P$):
\begin{itemize}
\item HAMILTON-KREIS
\item CLIQUE
\item SUBSET-SUM
\end{itemize}
\end{frame}
\end{document}