\documentclass{beamer}
\usepackage[utf8]{inputenc}
\usepackage[ngerman]{babel}

\usetheme[deutsch]{KIT}
\author{Simon Bischof (simon.bischof2@student.kit.edu)}
\title{Tutorium Theoretische Grundlagen der Informatik}
\subtitle{Simon Bischof}
\institute{Institut f\"{u}r Kryptographie und Sicherheit}
\TitleImage[scale=0.7]{tmaschine.png}

\begin{document}
\begin{frame}
\maketitle
\end{frame}

\begin{frame}
\frametitle{kontextsensitive Grammatiken}
\begin{itemize}
\item Einschränkung der Produktionen
\begin{itemize}
\item $u\to v$ mit $u\in V^+, v\in \left((V\cup T)\setminus \{S\}\right)^+$, $|u|\leq|v|$
\item oder $S\to\lambda$
\end{itemize} \pause
\item Äquivalent (siehe VL):
\begin{itemize}
\item $S\to\lambda$, $A\to a$ mit $A\in V$, $a\in\Sigma$
\item $\alpha A\beta\to\alpha\gamma\beta$ mit $A\in V,\quad \alpha,\beta,\gamma\in V^*,\quad \gamma\neq\lambda$
\end{itemize}
\end{itemize}
\end{frame}

\begin{frame}
\frametitle{linear beschränkte TM}
\begin{itemize}
\item arbeitet nur auf Platz, der von Eingabe belegt ist
\item eventuell noch auf dem einen Platz rechts davon
\end{itemize}
\end{frame}

\begin{frame}
\frametitle{Äquivalenzen}
\begin{itemize}
\item kontextsensitive Grammatiken $\Leftrightarrow$ \\(nicht-deterministische) linear beschränkte Turingmaschinen
\item Chomsky 0 $\Leftrightarrow$ Turingmaschinen
\end{itemize}
\end{frame}

\begin{frame}
\frametitle{Chomsky-Hierarchie}
\begin{itemize}
\item[] Chomsky 0 (TM)
\item[$\supsetneq$] entscheidbare Sprachen
\item[$\supsetneq$] Chomsky 1 (kontextsensitiv)
\item[$\supsetneq$] Chomsky 2 (kontextfrei)
\item[$\supsetneq$] Chomsky 3 (regulär)
\end{itemize}
\end{frame}

\begin{frame}
\frametitle{Universelle TM}
\begin{itemize}
\item jede Turingmaschine lässt sich codieren als Wort aus $\{0,1\}^*$ (Gödelnummer)
\vspace{1cm}\pause
\item $\exists$ Universelle TM T
\begin{itemize}
\item mit $\Sigma_T=\{0,1\}$ und $\Gamma_T=\Sigma\cup\{\square\}$
\item Eingabe: Gödelnummer von einer TM $M$, Wort $w\in\Sigma_M^*$
\item $T$ simuliert $M$ auf $w$
\end{itemize}
\end{itemize}
\end{frame}
\end{document}