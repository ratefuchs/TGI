\documentclass{beamer}
\usepackage[utf8]{inputenc}
\usepackage[ngerman]{babel}

\usetheme[deutsch]{KIT}
\author{Simon Bischof (simon.bischof2@student.kit.edu)}
\title{Tutorium Theoretische Grundlagen der Informatik}
\subtitle{Simon Bischof}
\institute{Institut f\"{u}r Kryptographie und Sicherheit}
\TitleImage[scale=0.7]{tmaschine.png}

\newcommand{\F}{\Sigma^*}

\begin{document}
\shorthandoff{"}
\begin{frame}
\maketitle
\end{frame}

\begin{frame}
\frametitle{Ergänzung zur Gödelnummer}
\begin{itemize}
\item für TM $M$ bezeichnet $\langle M\rangle$ die Gödelnummer von $M$
\item für $w\in\{0,1\}^*$ ist $M_w$ die TM mit Gödelnummer $w$ \pause
\item für nicht korrekte Gödelnummer $w$ ist $L(M_w)=\emptyset$
\end{itemize}
\end{frame}

\begin{frame}
\frametitle{(Semi-)Entscheidbarkeit}
Sei $L\subset\F$ eine Sprache.
\begin{itemize}
\item $L\in R$ ($L$ ist entscheidbar) $:\Leftrightarrow$ es existiert eine TM $M$, die $L$ entscheidet (d.h. $M$ hält immer) \pause
\item $L\in RE$ ($L$ ist semientscheidbar) $:\Leftrightarrow$ es existiert eine TM $M$, die $L$ akzeptiert (d.h. für $w\notin L$ muss $M$ nicht notwendigerweise halten)
\item $L\in co-RE :\Leftrightarrow \bar{L} := \F\setminus L\in RE$
\item $R=RE\cap co-RE$
\end{itemize}
\end{frame}

\begin{frame}
\frametitle{Nicht entscheidbare Sprachen / Probleme}
\begin{itemize}
\item Diagonalsprache $L_D:=\{w\in\F | M_w \text{ akzeptiert } w \text{ nicht}\}$
\item Wortproblem $A_{TM}:=\{\langle M\rangle w\in\F | M \text{ akzeptiert } w\}$
\item Halteproblem $HALT:=\{\langle M\rangle w\in\F | M \text{ hält bei Eingabe } w\}$
\item $MIN_{TM} := \{\langle M\rangle | M$ ist minimale TM$\}$.
\end{itemize}
\end{frame}

\begin{frame}
\frametitle{Berechenbarkeit, Many-One-Reduzierbarkeit}
\begin{itemize}
\item $f:\F\to\F$ heißt berechenbar, wenn eine TM $M$ exisiert, die bei Eingabe $w$ mit $f(w)$ auf dem Band hält.
\item $A\leq_m B$ ($A$ Many-One-reduzierbar auf $B$), wenn eine berechenbare Funktion $f:\F\to\F$ existiert mit $w\in A\Leftrightarrow f(w)\in B$
\end{itemize}
Sei $A\leq_m B$. Dann:
\begin{itemize}
\item $B$ entscheidbar $\Rightarrow$ $A$ entscheidbar.
\item $A$ nicht entscheidbar $\Rightarrow$ $B$ nicht entscheidbar
\end{itemize}
\end{frame}

\begin{frame}
\frametitle{Postsches Korrespondenzproblem}
Gegeben: eine endliche Menge von "Puzzlestücken"
$$S=\left\lbrace\begin{pmatrix}t_1\\b_1\end{pmatrix}, \ldots,\begin{pmatrix}t_n\\b_n\end{pmatrix}\right\rbrace$$ 
mit $t_1, \ldots, t_n, b_1, \ldots, b_n\in\F$.\\
Frage: Existieren $i_1, \ldots, i_k$ mit $b_{i_1}\ldots b_{i_k}=t_{i_1}\ldots t_{i_k}$?
\end{frame}

\begin{frame}
\frametitle{Quines}
\begin{itemize}
\item Sei $P_w$ die TM, die w ausgibt und hält.
\item Es gibt eine berechenbare Funktion $q:\F\to\F$ mit $q(w)=\langle P_w\rangle$\pause
\item Es existiert eine TM, die ihre eigene Gödelnummer ausgibt
\end{itemize}
\end{frame}

\begin{frame}
\frametitle{Das Rekursionstheorem}
\begin{itemize}
\item 1. Form: Die TM $T$ berechne die Funktion $t:\F\times\F\to\F$. Dann existiert eine TM $R$, die die Funktion $r:\F\to\F$ mit $r(w)=t(\langle R\rangle, w)$ berechnet.\pause
\item 2. Form: Es sei $t:\F\to\F$ berechenbar. Dann existiert eine TM $F$, so dass $M_{t(\langle F\rangle)}$ die gleiche Funktion berechnet wie $F$. 
\end{itemize}
\end{frame}
\end{document}