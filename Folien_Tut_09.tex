\documentclass{beamer}
\usepackage[utf8]{inputenc}
\usepackage[ngerman]{babel}

\usetheme[deutsch]{KIT}
\author{Simon Bischof (simon.bischof2@student.kit.edu)}
\title{Tutorium Theoretische Grundlagen der Informatik}
\subtitle{Simon Bischof}
\institute{Institut f\"{u}r Kryptographie und Sicherheit}
\TitleImage[scale=0.7]{tmaschine.png}

\newcommand{\F}{\Sigma^*}
\newcommand{\N}{\ensuremath \mathbb{N}}
\newcommand{\R}{\ensuremath \mathbb{R}}
\renewcommand{\P}{\ensuremath \mathcal{P}}
\newcommand{\NP}{\ensuremath \mathcal{NP}}
\newcommand{\NPC}{\ensuremath \mathcal{NP-C}}

\begin{document}
\shorthandoff{"}
\begin{frame}
\maketitle
\end{frame}

\begin{frame}
\frametitle{Zum Übungsblatt}
\begin{itemize}
\item Beim PKP ist eine leere Puzzlestückfolge KEINE gültige Lösung
\item Turingreduktion noch mal anschauen!
\item Wenn ihr Entscheidbarkeit bewiesen habt, folgt daraus automatisch Semientscheidbarkeit
\item $K(x)$: immer hinschreiben, dass $c$ konstante Größe der TM ist
\item Th$(\N,+)$: eigentlich nur Gleichungen der Form $x+y=z$ erlaubt
\end{itemize}
\end{frame}

\begin{frame}
\frametitle{$\NP$-Vollständigkeit}
\end{frame}

\begin{frame}
\frametitle{polynomielle many-one-Reduktion}
\begin{itemize}
\item Sei $f$ in polynomieller Zeit berechenbare Funktion und $A$ und $B$ Sprachen
\item Sei $\forall w\in\F: w\in A\Leftrightarrow f(w)\in B$
\item Dann ist $A$ polynomiell many-one-reduzierbar auf $B$ ($A\leq_p B$)\pause
\item $A\leq_p B$ und $B\in\P$ $\Rightarrow$ $A\in\P$
\item poly many-one-Reduzierbarkeit ist transitiv\pause
\item poly many-one-reduzierbar ist nicht dasselbe wie poly turingreduzierbar
\end{itemize}
\end{frame}

\begin{frame}
\frametitle{$\NP$-Schwere und -Vollständigkeit}
\begin{itemize}
\item $A$ ist $\NP$-schwer, falls $\forall B\in\NP: B\leq_p A$\pause
\item $A$ ist $NP$-vollständig ($A\in\NPC$), falls $A\in\NP$ und $A$ $\NP$-schwer ist\pause
\item $B\in\NPC$, $A\in\NP$, $B\leq_p A$ $\Rightarrow$ $A\in\NPC$\pause
\item falls $\P\cap\NPC\neq\emptyset$, ist $\P=\NP$
\end{itemize}
\end{frame}

\begin{frame}
\frametitle{SAT (Erfüllbarkeit)}
\begin{itemize}
\item SAT $:=\{\text{boolsche Formel } b|$\\$\text{es existiert eine Variablenbelegung, so dass $b$ wahr wird}\}$ (Version 1)
\item meist eingeschränkt auf konjunktive Form: z.B.\\ $b=(\underbrace{x_1}_\text{Literal}\vee x_3\vee\underbrace{\bar{x}_4}_\text{Literal}\vee x_5)\wedge\underbrace{(\bar{x}_1\vee x_2)}_\text{Klausel}\wedge (x_4)$\\ $x_1,\ldots, x_n$ nennt man Variablen\pause
\item Satz von Cook: $SAT\in\NPC$\pause
\item Erfüllbarkeit von disjunktiven Formen ist aber in $\P$
\end{itemize}
\end{frame}

\begin{frame}
\frametitle{weitere wichtige $\NP$-vollständigen Probleme}
\begin{itemize}
\item siehe Tutblatt
\item siehe Übungsblätter
\item siehe Vorlesungsfolien und Skript von Wagner (http://i11www.iti.uni-karlsruhe.de/teaching/winter2011/tgi/index) \\{} [für diese VL natürlich inoffiziell...]
\end{itemize}
\end{frame}

\begin{frame}
\frametitle{Zero-Knowledge-Beweise}
\end{frame}

\begin{frame}
\frametitle{Ziel von ZK-Beweisen}
Einen anderen überzeugen, eine Lösung zu kennen, ohne
\begin{itemize}
\item diese zu verraten
\item dass andere von einer Mitschrift des "Gesagten" überzeugt werden
\end{itemize}
\end{frame}

\begin{frame}
\frametitle{Wie kann so etwas gehen?}
\begin{itemize}
\item mithilfe von Zufall: falsche Lösungen werden "ziemlich sicher" erkannt\pause
\item verwende eine Art "Zeuge", der aber nur teilweise abgefragt wird\pause
\item zuerst wählt der Beweiser den "Zeugen", dann sagt der Prüfer, welchen Teil er wissen will\pause
\item Wiederholung des obigen Schritts bis die Fehlerwahrscheinlichkeit "gering genug"\pause
\item wichtig dabei: falls der Beweiser wüsste, welchen Teil der Prüfer sehen will, würde er den Zeugen entsprechend "fälschen" können (sieht daher für außenstehende wie Absprache aus)
\end{itemize}
\end{frame}

\begin{frame}
\frametitle{ZK-Beweise und $\NP$}
\begin{itemize}
\item 3-COLOR $\in\NPC$
\item 3-COLOR lässt sich ZK-beweisen
\item alle Probleme aus $\NP$ lassen sich ZK-beweisen
\end{itemize}
\end{frame}
\end{document}