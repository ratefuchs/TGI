\documentclass{beamer}
\usepackage[utf8]{inputenc}
\usepackage[ngerman]{babel}

\usetheme[deutsch]{KIT}
\author{Simon Bischof (simon.bischof2@student.kit.edu)}
\title{Tutorium Theoretische Grundlagen der Informatik}
\subtitle{Simon Bischof}
\institute{Institut f\"{u}r Kryptographie und Sicherheit}
\TitleImage[scale=0.7]{tmaschine.png}

\newcommand{\F}{\Sigma^*}
\newcommand{\N}{\ensuremath \mathbb{N}}
\newcommand{\R}{\ensuremath \mathbb{R}}

\begin{document}
\shorthandoff{"}
\begin{frame}
\maketitle
\end{frame}

\begin{frame}
\frametitle{Zum Übungsblatt}
\begin{itemize}
\item beim "Simulieren" alle Konfigurationen angeben, außer es steht explizit was anderes da
\item Zwischenschritte beim Umformen in Chomsky-NF machen es dem Tutor einfacher
\end{itemize}
\end{frame}

\begin{frame}
\frametitle{Gödels Unvollständigkeitssatz}
\begin{itemize}
\item Jedes "hinreichend mächtige" formale System ist entweder widersprüchlich oder unvollständig.\pause
\item Bsp. für hinreichend mächtig: $\N$ mit $+$ und $*$ (Th$(\N,+,*)$)
\end{itemize}
\end{frame}

\begin{frame}
\frametitle{(Nicht-)Entscheidbarkeit von wichtigen Theorien}
\begin{itemize}
\item Th$(\N,+)$ ist entscheidbar.\pause
\vspace{1cm}
\item Th$(\N,+,*)$ ist unentscheidbar.
\end{itemize}
\end{frame}

\begin{frame}
\frametitle{Was ist ein Beweis?}
\begin{itemize}
\item Ein Beweis ist (maschinen-)überprüfbar.
\item Alle beweisbaren Aussagen sind wahr ("Soundness").\pause
\item Die Menge der beweisbaren Aussagen in Th$(\N,+,*)$ ist rekursiv aufzählbar.
\item[$\Rightarrow$] Es existieren nicht beweisbare Aussagen in Th$(\N,+,*)$.\pause
\item Für jedes Kalkül (mit "Soundness" und Turingentscheidbarkeit der Gültigkeit von Ableitungen) gibt es eine Aussage, die im Kalkül nicht beweisbar ist.
\end{itemize}
\end{frame}

\begin{frame}
\frametitle{Turingreduzierbarkeit}
\begin{itemize}
\item Ein Orakel für eine Sprache L ist ein "externes Gerät", das als Hilfe für eine TM entscheidet, ob ein Wort $w\in L$ ist.
\item $TM^O:=$ Turingmaschine mit Zugriff auf Orakel $O$.\pause
\item $A \leq_T B :=$ es existiert eine Orakel-TM $TM^O$, die $A$ entscheidet, wobei $O$ Orakel für $B$ (Turingreduzierbarkeit).\pause
\item $A \leq_T B$ und $B$ entscheidbar $\Rightarrow A$ entscheidbar
\item Halteproblem für TM mit Orakel $O$ nicht durch Turingmaschinen mit Orakel $O$ entscheidbar.
\end{itemize}
\end{frame}

\begin{frame}
\frametitle{ein kleiner Versuch...}
\only<2>{374986932084149032749124709129269196895327\\
8741974981629846071486258321418884}
\only<4>{012345678910111213141516171819202122232425\\
2627282930313233343536373839404142}
\only<5>{}
\end{frame}

\begin{frame}
\frametitle{Kolmogorow-Komplexität}
\begin{itemize}
\item stelle Wort $w$ durch $\langle M\rangle 01w^\prime$ dar, wobei
\begin{itemize}
\item $01$ ist "Trennzeichen"
\item $M$ bei Eingabe $w^\prime$ hält und $w$ aufs Band schreibt
\end{itemize}\pause
\item $K(w)$ ist die Länge der kürzesten Codierung für $w\in\{0,1\}^*$ nach obiger Form (Kolmogorow-Komplexität)
\end{itemize}
\end{frame}
\end{document}