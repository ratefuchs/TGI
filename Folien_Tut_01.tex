\documentclass{beamer}
\usepackage[utf8]{inputenc}
\usepackage[ngerman]{babel}

\usetheme[deutsch]{KIT}
\author{Simon Bischof (simon.bischof2@student.kit.edu)}
\title{Tutorium Theoretische Grundlagen der Informatik}
\subtitle{Simon Bischof}
\institute{Institut f\"{u}r Kryptographie und Sicherheit}
\TitleImage[scale=0.7]{tmaschine.png}

\begin{document}
\begin{frame}
\maketitle
\end{frame}

\begin{frame}
\frametitle{Organisatorisches}
\end{frame}

\begin{frame}
\frametitle{Vorstellung}
Zu meiner Person:\pause
\begin{itemize}
\item Simon Bischof
\item Bachelor Informatik
\item 5. Semester
\end{itemize}
\end{frame}

\begin{frame}
\frametitle{Kontakt}
\begin{itemize}
\item E-Mail: simon.bischof2@student.kit.edu \pause
\item bei inhaltlichen und sonstigen Fragen zu TGI \pause
\item Liste zum Eintragen der Mailadresse
\item Materialien, kurzfristige Informationen
\end{itemize}
\end{frame}

\begin{frame}
\frametitle{Übungsbetrieb}
Es gibt einen Übungsschein:
\begin{itemize}
\item Übungsblattabgabe in Fünfergruppen
\item auch über Tutoriumsgrenzen hinweg \pause
\item \textbf{Wechsel in andere Gruppen nicht möglich!} \pause
\item Festlegung der Gruppe mit erster Übungsblattabgabe \pause
\item jeder gibt ein Blatt ab (handschriftlich)
\end{itemize}
\end{frame}

\begin{frame}
\frametitle{Übungsbetrieb - Fortsetzung}
\begin{itemize}
\item in allen Blättern außer einem jeweils $\geq$ 50\% der maximalen Punkte
\item $\Rightarrow$ ein Notenschritt Bonus auf bestandene Klausur \pause
\item Übungsblätter sind gute Übung für Klausur
\item $\Rightarrow$ Blätter auch unabhängig vom Schein machen \pause
\item achtet bitte auf formale Korrektheit
\end{itemize}
\end{frame}

\begin{frame}
\frametitle{zum Tutorium}
\begin{itemize}
\item Stoff soll wiederholt werden \pause
\item Dabei Fokus auf Übungsbetrieb \pause
\item Fragen/Vorschläge/Anmerkungen willkommen!
\end{itemize}
\end{frame}

\begin{frame}
\frametitle{nächste Woche}
\begin{itemize}
\item Do., 1.11., ist Feiertag \pause
\item Tutorium am Freitag findet trotzdem statt \pause
\item Wer da nicht kann: bitte in ein anderes Tut gehen
\end{itemize}
\end{frame}

\end{document}